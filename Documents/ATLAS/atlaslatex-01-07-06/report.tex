\documentclass[a4paper,12pt, onecolumn]{article}
  \begin{document}

  The uncertainty principle states that
  \begin{equation}
    \Delta x \Delta p \geq \frac{\hbar}{2}
   \end{equation}
  where $\Delta x$ represents the uncertainty in a measurement of position, and 
  $\Delta p$ represents uncertainty in measurement of momentum. 
  (momentum = mass x speed) \\

  The uncertainty principle tells us that if one knows the exact position of a particle
  \footnote{By particle I mean anything really. You could think of them as say molecules
    in a gas. The movement of molecules in a gas will determine its properties}
  (hence $\Delta x$ is very small) then the uncertainty in the momentum of a 
  particle becomes very large. Hence, one \emph{cannot} know both the position 
  and momentum of a particle at the same time. Therefore one cannot determine 
  exactly how the particle will move, as one needs to know both its momentum 
  and its position to achieve this. \\

  Determinism (as I loosely understand it) says that, if we were clever enough, we 
  could measure the position and momentum of every particle in the world and then from
  that we could calculate how every particle would move and therefore everything that 
  would happen in the world, including all our decisions. \\

  However, the uncertainty principle shows this to be untrue as we cannot know the 
  position and momentum of every particle in the universe. Nature has an intinsic 
  unpredictability about it.

  \end{document}
